\def\exercise #1:{\bigskip\noindent{\bf EXERCISE #1}\par\nobreak\medskip}

% ------------------------------------------------------------

Lorem ipsum dolor sit amet, consectetur adipiscing elit.
Vivamus enim sapien, ornare tempor posuere sed, pharetra quis ante.
Sed rutrum nunc non aliquam dapibus.
Nulla facilisi.
Donec vestibulum volutpat lectus vel convallis.
Donec aliquet libero ut elementum fermentum.
Nam dapibus mattis ex, vel elementum risus elementum vitae.
Donec tincidunt, dolor nec efficitur dapibus, nulla augue suscipit elit, et porttitor sapien lectus at metus.
Curabitur erat turpis, rhoncus in leo in, tempus sodales nibh.
Proin tristique a ex quis venenatis.
Sed porttitor dolor a leo interdum, sit amet dapibus risus elementum.

{\it\narrower\narrower\smallskip\noindent
This paragraph will have narrower lines than
the surrounding paragraphs do, because it uses the ``narrower'' feature of plain \TeX.
The former margins will be restored after the group ends.\smallskip}
Nam vel tortor odio.
Nam tempor ipsum eu urna viverra, sit amet tincidunt eros ornare.
Maecenas ligula magna, maximus at eros non, mattis commodo erat.
Donec fringilla nisi tellus, id blandit est suscipit et.
Sed hendrerit imperdiet sapien et commodo.
Aliquam placerat ac velit quis feugiat.
Vivamus dapibus nunc vel mi euismod maximus.
Etiam in arcu non mi gravida dignissim sit amet eu sem.
Suspendisse ornare turpis eu dui maximus, sed euismod nibh pulvinar.
Phasellus lorem quam, interdum varius pretium et, ornare sit amet mauris.

\parindent0pt

Example of list:
\item{1.} This is the first of several cases that are being
enumerated, with hanging indentation applied to entire paragraphs.
\itemitem{a)} This is the first subcase.
\itemitem{b)} And this is the second subcase. Notice
that subcases have twice as much hanging indentation.
\item{2.} The second case is similar.


Leaders:
\def\leaderfill{\leaders\hbox to 1em{\hss.\hss}\hfill}

\line{Alpha\leaderfill Omega}
\line{TheBeginning\leaderfill The Ending}

Font sizes:

\font\ninerm=cmr9
\font\eightrm=cmr8
\font\sixrm=cmr6

For example: \tenrm smaller \ninerm and smaller \eightrm and smaller \sevenrm and smaller \sixrm and smaller \fiverm and smaller \tenrm, and now back to normal.

\font\magnifiedfiverm=cmr5 at 10pt
Ten-point type is different from {\magnifiedfiverm magnified five-point type}.

\font\onebigtenrm=cmr10 scaled\magstep1
\font\twobigtenrm=cmr10 scaled\magstep2
\font\onebigtentt=cmtt10 scaled\magstep1
\font\twobigtentt=cmtt10 scaled\magstep2

This is {\tentt cmr10} at normalsize.

\onebigtenrm This is {\onebigtentt cmr10} scaled once by 1.2.

\twobigtenrm This is {\twobigtentt cmr10} scaled twice once by 1.2.

\tenrm

% ------------------------------------------------------------------------


\bigskip
\hrule
\bigskip

\exercise 17.17:

\def\euler{\atopwithdelims\langle\rangle}

$$
a = {n \euler k}
$$

\exercise 21.3:

\def\boxit#1{\vbox{\hrule\hbox{\vrule\kern3pt
   \vbox{\kern3pt#1\kern3pt}\kern3pt\vrule}\hrule}}

\setbox4=\vbox{\hsize 23pc \strut For example, the sentence you are now reading...\strut}

$$
\boxit{\boxit{\box4}}
$$


\bye